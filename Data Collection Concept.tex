\documentclass {article}

\begin{document}

\title{THE INFLUENCE OF SMARTPHONES ON LEARNING AMONG UNIVERSITY STUDENTS}
\author{KOMAKECH RONALD 15/u/6690/EVE 215011576}

\maketitle

\section{Introduction}
Nowadays, Smartphones turn out to be a major part of our life due to their advanced features. It is difficult to avoid such new technologies much as we know the effect they have on our society as well as the environment. Despite the many fields impacted upon by the invention of Smartphones, this study focuses on the education field and it is intended to help me understand all the positive and negative aspects of Smartphones on learning at university. The study will primarily focus on students at higher institutions of learning namely university.




\subsection{Background}

The first telephone was invented by Alexander Graham Bell and it was not until 1878 when he made the first phone call.  Since then, telephones have not only come a long way, but may one day be completely obsolete. 
This is due to the rapid advancement in technology that has seen the telephone evolve from a mere device meant to receive calls and send messages to Smartphones.
The first smartphone was developed by IBM and BellSouth in 1993. Although basic compared to today’s standards “Simon” had a touch screen that was capable of accessing email and sending faxes. 
Smartphones are major extensions on normal cellphones. Cellphones can make phone calls and even some have video recording capabilities but they do not have GPS capabilities along with a whole array of other applications. 
Smartphones capability does not end at the Internet access, or at document editing. Smartphones also have the ability to interpret and decipher information like that from a quick response code that may be on a product’s packaging. Smartphone users can download QR code scanners as well as other applications so they have the ability to read the information embedded in the QR code that may take them to a website, a coupon, or even a social media site.



\subsection{Problem Statement}
The invention of Smartphones meant that phones could not only be used to make calls and receive but they could also perform a lot more functions including connection to the internet. This is primarily the sole reason for the popularity of Smartphones coupled with the development of mobile applications such as WhatsApp, Facebook, twitter etc.
These common mobile applications which come pre-installed on most Smartphones have risen concern among different scholars who question the benefits of Smartphones in the field of education. This is due to the fact that most students especially at university use their phones for mostly social networking instead of research.
Therefore this research is intended to bring to a compromise the two extremes i.e. the negatives and the positives of smartphones such that adequate time can be allocated to both research work and leisure. This will be through seeking the views of different students on how best to reduce the time spent on social media.


\subsection{Objectives}

\subsubsection{Main Objectives}

To determine the impact of smartphones on learning of university students.
\subsubsection{Specific Objectives}
To collect all the data necessary to aid my research.
To analyze the collected data.
To come up with a conclusion from the analyzed data.



\subsection{Scope}
This research basically focuses on the impact of smartphones in the education field and the data collected is mainly from students in the various universities in Uganda.

\section{Literature Review}
Different  research scholars have interesting opinions as regards the impact brought about by Smartphones in the education sector as listed below.
In  their  research  titled  “Smartphone  Addiction  in University Students and Its Implication for Learning,  Lee  found  that  the  higher  the addiction  level  is,  the  lower  level  of  self -regulated learning the students have, as well as low level of flow when   studying.   Further   interview   for   smartphone addiction group was conducted, it has been found that the    smartphone    addict—learners    are    constantly interrupted  by  the  other  applications  on  the  phones when  they  are  studying,  and  does  not  have  enough control  over  their  smartphone  learning  plan  and  its process.
Grosseck and  in their study found that the  majority  of  students  spend  significant  time  on Facebook  more  for  social  uses  (to  stay  in  touch  with friends and family, to share / tag photos, to engage in social   activism,   volunteering   etc.)   and   less   for academic   purposes,   even   if   they   take   part   in discussions  about  their  assignments,  lectures,  study notes  or  share  information  about research  resources etc.
In    their    research    concerning    Online    Social networking   (OSN)   Paul   their   results 
revealed a statistically significant negative relationship between  time  spent  by  students  on  OSN  and  their academic  performance.  The  time  spent  on  OSN  was found to be heavily influenced by the attention span of the  students.  Specifically,  we  determined  that  the higher  the  attention  span,  the  lower  is  the  time  spent on  OSN.  Further,  attention  span  was  found  to  be highly  correlated  with  characteristics  that  predict  or influence  student  behavior,  such  as  their  perceptions about society’s view of social networking, their likes and dislikes of OSN, ease of use of OSN, etc.




\section{Methodology}
The proposed methodology consists of two phases, data collection and data analysis.
Data  will  be  collected  using  ODK  Collect,  which  will  later  on  be  uploaded  to  the  ODK
aggregate  server  to  carry  out  all  the  required  analysis. Students at different universities will be approached and asked about their views on the topic above and these will be captured using ODK collect.
Among the data to be captured for each student will be;
Name of student, university attended, photo of interviewee, GPS coordinates etc.
When data from a reasonable number of students is collected, it will be analyzed and a conclusion will be drawn as well as recommendations on how to mitigate the negatives of smartphones in the education field.



\nocite{*}
\bibliographystyle{plain}
\bibliography{bibtex}



\end{document}