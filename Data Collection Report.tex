\documentclass {article}
\usepackage{graphicx}
\usepackage{subcaption}
\usepackage{datetime}
\usepackage{placeins}
\newdate{date}{25}{02}{2018}
\date{\displaydate{date}}

\begin{document}

\title{THE INFLUENCE OF SMARTPHONES ON LEARNING AMONG UNIVERSITY STUDENTS}
\author{KOMAKECH RONALD 15/u/6690/EVE 215011576}

\maketitle

\section{Executive summary}

The aim of this study was to determine how effectively students at the various universities in the country utilize their smartphones as far as their education lifestyle is concerned. The study also highlights the setbacks brought about by misuse of Smartphones by university students and how these setbacks can be mitigated.

\section{Introduction}
A Smartphone is a mobile phone with advanced services of communication and computing. However, despite its high-tech functions and advantages, with its increasing popularity the smartphone has negatively influenced individuals and society as a whole. So in this research, emphasis is put on how these devices of high-tech functions have influenced university students within the country.


\section{Methodology}
The information used in this research was collected from students at the different universities in the country. These were approached and requested to fill in questions contained in the ODK form and upon completion of the form it was uploaded onto the ODK aggregate server. Below is a sample of the sections of the form that were filled in by one of the interviewees. 
\newpage
\begin{figure}
  \begin{subfigure}[b]{0.5\textwidth}
    \includegraphics[width=\textwidth]{form.png}
    \caption{odk collect form}
    \label{fig:1}
  \end{subfigure}
  %
  \begin{subfigure}[b]{0.5\textwidth}
    \includegraphics[width=\textwidth]{Name.png}
    \caption{Name of the interviewee}
    \label{fig:2}
  \end{subfigure}
\end{figure}
\FloatBarrier

\begin{figure}
  \begin{subfigure}[b]{0.5\textwidth}
    \includegraphics[width=\textwidth]{pic.png}
    \caption{Photo of the interviewee}
    \label{fig:1}
  \end{subfigure}
  %
  \begin{subfigure}[b]{0.5\textwidth}
    \includegraphics[width=\textwidth]{data.png}
    \caption{Data collected from an interviewee}
    \label{fig:2}
  \end{subfigure}
\end{figure}
\FloatBarrier


\section{Findings}
The data collected was visualized in form of a pie chart to show the distribution of Smartphones among students of different courses in one of the universities in the country. 

\begin{figure}[hb]
  \includegraphics[width=20cm]{pie.png}
  \caption{Pie chart showing the distribution of Smartphones among students of different courses.}
  
\end{figure}
\FloatBarrier


\section{Interpretation of findings}
 After analysing the data, a conclusion was drawn that no matter which course one offers, the need for a smartphone is equally important for each student at the university. Whether the students use it for academic purposes or not, It remains up to the universities to ensure that students use their smartphone effectively by establishing boundaries.
\section{Conclusion}
From the information examined during this research, it is apparent from above facts that  the  benefits  of  Smartphone  are  tremendous  and  negative impacts  are  minor.  So  it  is  important  to  concentrate  on  how  to  stop  and  avoid  smartly  the  misuse  of  Smartphone  rather  trying  to  stop  or  avoid  use to Smartphones. 
\section{Recommendations}

Since instead of doing away with smartphones intensive research can be done on how to combat the negative effects of smartphones. For example there are several initiatives from different  vendors that I would recommend universities within the country to employ in order to combat  the   misuse of  Smartphone at workplace and at  Universities.  These include ; SAP,  Airwatch,  MacAfee  and many other vendors  and their main objective is to provide  solutions  to  control  the  access of     Smartphones within the workplace and Universities.  

\end{document}